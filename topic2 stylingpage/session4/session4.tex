\documentclass{article}
\title{Creating first document in \LaTeX}
\author{wang yi}
\date{\today}

\begin{document}
    \maketitle
    \tableofcontents
    \section{Overview}
    \section{Introduction}
    \paragraph{}
    Dhansak is a popular Indian dish, originating among the Parsi Zoroastrian community.[1] It combines elements of Persian and Gujarati cuisine. Dhansak is made by cooking mutton or goat meat with a mixture of lentils and vegetables. This is served with caramelised brown rice, which is rice cooked in caramel water to give it a typical taste and colour. The dal cooked with mutton and vegetables served with brown rice, altogether is called dhansak.
    \subsection{Purpose}
    \subsection{Scope}
\end{document}